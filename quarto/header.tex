\usepackage{luatexja}
\usepackage{luatexja-fontspec}
\usepackage[a4paper,margin=20mm]{geometry}

% 日本語フォント(本文)
\setmainjfont{Noto Sans CJK JP}
% 英文フォント(必要なら)
\setmainfont{Noto Sans}
\usepackage{fontspec}

\directlua{
  luaotfload.add_fallback("emoji-fallback",{
    "NotoColorEmoji:mode=harf;",
    "Apple Color Emoji:mode=harf;",
    "Noto Emoji:mode=harf;",
  })
}
% 絵文字フォント(白黒で安定)
% macOSなら Symbola が無いこともあるので、あるフォント名に合わせて変更
% \newfontfamily\emojifont{Symbola}

% 代表的な記号を絵文字フォントに割り当て(必要な分だけ追加)
\DeclareTextFontCommand{\textemoji}{\emojifont}

\setmainfont{Noto Sans}[RawFeature={fallback=emoji-fallback}]
\setmainjfont{Noto Sans CJK JP}[RawFeature={fallback=emoji-fallback}]